\newcolumntype{L}[1]{
    >{
        \centering\arraybackslash
    }m{#1}
}


\begin{tabular}{
    L{0.15\linewidth}
    L{0.15\linewidth}
    L{0.35\linewidth}
    L{0.35\linewidth}
}
    \hline
    \hline
    Cambios en los parámetros & Propiedad de la solución afectada & Procedimiento para reoptimizar desde el primal & Procedimiento para reoptimizar desde el dual \bigstrut\\
    \hline
    \hline
    Cambio en $c_j$ & Óptimalidad & Debemos recalcular el renglón $\zero$.   En caso de que no sea óptimo, seguimos iterando & Verificamos si la solución Óptima actual verifica la restricción modificada.  En caso de no hacerlo es necesario modificar la tabla y trabajar sobre el primal asociado \bigstrut\\
    \hline
    Cambio en $b_i$ & Factibilidad & Verificamos si la solución Óptima cumple con la restricción modificada.  En caso de que no sea así modificamos la tabla y trabajamos sobre el problema dual & Verificamos si la solución Óptima lo sigue siendo (analizando el renglón $\zero$).   Caso contrario re iteramos. \bigstrut\\
    \hline
    Cambio en $a_{ij}$ no basica & Óptimalidad & Recalculamos la columna $A_j$ correspondiente a la $a_{ij}$ que se modifico.  Verificamos si la solución Óptima lo sigue siendo (analizando el renglón $\zero$) y en caso contrario re iteramos. & Recalculamos la columna $A_j$ correspondiente a la $a_{ij}$ que se modifico.  Verificamos si la solución Óptima lo sigue siendo (analizando el renglón $\zero$) y en caso contrario re iteramos. \bigstrut\\
    \hline
    Cambio en $a_{ij}$ basica & Factibilidad & Debemos trabajar sobre el problema dual asociado & Recalculamos la columna $A_j$ correspondiente a la $a_{ij}$ que se modifico.   Verificamos si la solución Óptima lo sigue siendo (analizando el renglón $\zero$) y en caso contrario re iteramos. \bigstrut\\
    \hline
    Agregado de una nueva Actividad & Óptimalidad & Agregamos una nueva columna a la tabla Óptima.  Verificamos si la solución Óptima lo sigue siendo (analizando el renglón $\zero$) y en caso contrario re iteramos. & La nueva actividad se convierte en una nueva restricción.   La re optimización es análoga al caso de nueva  restricción en el primal \bigstrut\\
    \hline
    Agregado de una nueva restricción & Factibilidad & Verificamos si la solución Óptima cumple con la restricción modificada.   En caso de que no sea así modificamos la tabla y trabajamos sobre el problema dual & La nueva restricción se convierte en una nueva actividad.    La re optimización es análoga al caso de nueva  actividad  en el primal \bigstrut\\
    \hline
    \hline
    \end{tabular}%
