

 
\begin{homeworkProblem}[-1][Gem de Vivian]
\subsubsection{Modelo Programación Lineal}

Las variables de decisión son:
\begin{itemize}
    \item $X_1$: Cantidad de joyas tipo 1 producidas. \\
    \item $X_2$: Cantidad de joyas tipo 2 producidas. 
\end{itemize}
La función objetivo:
\begin{align*}
    Max\ Z = \paren{10-5}\fraccorche{\$}{T_1}X_1\corche{T_1} + \paren{6-4}\fraccorche{\$}{T_2}X_2\corche{T_2}
\end{align*}
Sujeta a las siguientes restricciones:
\begin{align*}
    2&X_1 + 1X_2 \le 30 \\
    4&X_1 + 1X_2 \le 50 \\
    1&X_1 + 0X_2 \ge 11 \\
    &X_1,X_2 \ge 0
\end{align*}
Forma estándar:
\begin{align*}
    Max\ Z = 5&X_1 + 2X_2 + 0X_3 + 0X_4 + 0X_5 - M\mu_1 \\
    s.a.: 2&X_1 + 1X_2 + 1X_3 + 0X_4 + 0X_5 + 0\mu_1 = 30 \\
          4&X_1 + 1X_2 + 0X_3 + 1X_4 + 0X_5 + 0\mu_1 = 50 \\
          1&X_1 + 0X_2 + 0X_3 + 0X_4 - 1X_5 + 1\mu_1 = 11
\end{align*}
\insertarTabla{02.09.01.tex}{Solución Optima Primal}
La tabla anterior nos indica que se deben producir 11 joyas del tipo 1 y 6 del tipo 2. El recurso militante en este caso serán los diamantes, sobrando 2 rubíes. Si consiguiéramos un diamante mas nuestra ganancia aumentaría en \$2
\subsubsection{46 Diamantes}
Modelo Dual:
\begin{align*}
    Min\ W = 30&Y_1 + 50Y_2 + 11Y_3 \\
    s.a.: 2&Y_1 + 4Y_2 + 1Y_3 - 1Y_4 + 0Y_5 + 1\mu_1 + 0\mu_2 = 5 \\ 
     1&Y_1 + 1Y_2 + 0Y_3 + 0Y_4 - 1Y_5 + 0\mu_1 + 1\mu_2 = 2 \\ 
     &Y_1,Y_2 \ge 0 \\ -&Y_3 \ge 0
\end{align*}
\insertarTabla{02.09.02.tex}{Tabla Dual Optima}
Rango de variación de $b_2$:
\llavesR{
    -11\times \paren{-2} -  1b_2 \le 0 \\
    -11\times \paren{-4} - 1b_2 \le 0 
}{
    b_2 \ge 22 \wedge b_2 \ge 44
}
\begin{align*}
    B^{-1}.b_n = b^{*} \\
    \begin{bmatrix}
        1 & -1 & 2 \\
        0 & 1 & -4 \\
        0 & 0 & 1 
    \end{bmatrix}.
    \begin{bmatrix}
        30 \\
        46 \\
        11
    \end{bmatrix}=
    \begin{bmatrix}
        6 \\
        2\\
        11
    \end{bmatrix}
\end{align*}
Podemos ver que aun con la variación en la cantidad de diamantes, la solución sigue siendo tanto optima como factible, siendo la nueva solución:
\insertarTabla{02.09.03.tex}{Nueva solución Optima 01}

\subsubsection{12 joyas \RN{1}}
Intervalo de variación de $c_2$
\llavesR{
    5\times 0 + c_2 \ge 0 \\
    5\times \paren{-1} + 4c_2 \ge 0
}{
    c_2 \ge 0 \wedge c_2 \ge 1.25
}
\begin{align*}
B^{-1}.c_k=c_k^{*}\\
    \begin{bmatrix}
        0 & 1 \\
        -1 & 4
    \end{bmatrix}.
    \begin{bmatrix}
        5 \\
        1.5        
    \end{bmatrix}=
    \begin{bmatrix}
        1.5 \\
        1
    \end{bmatrix}
\end{align*}
\insertarTabla{02.09.04.tex}{Nueva Solución Optima 02}
Podemos ver que la solución nuevamente es tanto optima como factible, arrojando un resultado de 45 en este caso.

\subsubsection{Nueva Joya}
Rango de variación de $b_3$.
\llavesR{
    -50 -2b_3 \le 0 \\
    -50 -4b_3 \le 0
}{
    b_3 \ge -2.5 \wedge b_3 \ge -12.5
}
\begin{align*}
    B^{-1}.b_n = b^{*} \\
    \begin{bmatrix}
        1 & -1 & 2 \\
        0 & 1 & -4 \\
        0 & 0 & 1 
    \end{bmatrix}.
    \begin{bmatrix}
        30 \\
        50 \\
        12
    \end{bmatrix}=
    \begin{bmatrix}
        4 \\
        2\\
        12
    \end{bmatrix}
\end{align*}
En este caso la solución también sigue siendo optima y factible, siendo esta:
\insertarTabla{02.09.05.tex}{Nueva Solución Optima 03}
\end{homeworkProblem}

