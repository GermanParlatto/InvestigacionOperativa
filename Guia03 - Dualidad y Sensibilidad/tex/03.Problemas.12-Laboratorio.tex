\begin{homeworkProblem}[-1][Laboratorio]
\subsubsection{Plan de Producción}
La función objetivo es:
\begin{align*}
    Max\ Z = 10\fraccorche{\$}{u1}X_1\corche{u1} +                           20\fraccorche{\$}{u2}X_2\corche{u2} +                           40\fraccorche{\$}{u3}X_3\corche{u3} +                           32\fraccorche{\$}{u4}X_4\corche{u4}   
\end{align*}
Sujeta a:
\begin{align*}
    10\fraccorche{m^2}{u1}X_1\corche{u1} +                           30\fraccorche{m^2}{u2}X_2\corche{u2} +                           80\fraccorche{m^2}{u3}X_3\corche{u3} +                           42\fraccorche{m^2}{u4}X_4\corche{u4} \le 900\corche{m^2}\\
    2\fraccorche{T}{u1}X_1\corche{u1} +                          1\fraccorche{T}{u2}X_2\corche{u2} +                           1\fraccorche{T}{u3}X_3\corche{u3} +                           3\fraccorche{T}{u4}X_4\corche{u4} \le 80\corche{T}
\end{align*}
\insertarTabla{02.12.01.tex}{Solución Óptima}
\subsubsection{Informe}
Para poder maximizar nuestras ganancias debemos producir 20 unidades del producto 4 y 10 unidades del producto 1, produciendo 0 unidades tanto del segundo como tercer producto. De esta manera alcanzamos un beneficio de $\$740$ consumiendo la totalidad de los recursos.
\subsubsection{Más almacén}
\begin{align*}
B^{-1}.b_n=b^{*}\\
    \begin{bmatrix}
        0.04 & -0.20 \\
        -0.06 & 0.80
    \end{bmatrix}.
    \begin{bmatrix}
        1050 \\
        80        
    \end{bmatrix}=
    \begin{bmatrix}
        26 \\
        1
    \end{bmatrix}
\end{align*}
Por lo que se ve anteriormente, la solución continua siendo factible.
\insertarTabla{02.12.02.tex}{Nueva Solución Óptima}
Obtenemos una ganancia final de $\$842$, siendo la diferencia con el plan anterior de $\$102$, lo que se traduce en un beneficio final de $\$32$, por lo que resulta beneficioso la contratación del espacio extra.
\subsubsection{Nuevo Producto}
\begin{align*}
A^{*}=B^{-1}.A_7\\
    \begin{bmatrix}
        0.04 & -0.20 \\
        -0.06 & 0.8
    \end{bmatrix}.
    \begin{bmatrix}
        20 \\
        2        
    \end{bmatrix}=
    \begin{bmatrix}
        0.4 \\
        0.4
    \end{bmatrix}
\end{align*}
La solución nueva sigue siendo tanto Óptima como factible, sin embargo, el renglón \zero para el nuevo producto es mayor a $0$, por lo que al tratarse de un problema de maximizacion nos indica la no conveniencia de la actividad que representa al mismo
\end{homeworkProblem}