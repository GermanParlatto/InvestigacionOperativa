
\subsubsection{Solución e Interpretación}
\begin{align*}
    Max Z = 300\fraccorche{\$}{Pe}X_1\corche{Pe} + 200\fraccorche{\$}{Pa}X_2\corche{Pa} \\
\end{align*}
\begin{align*}
    6\fraccorche{e}{Pe}X_1\corche{Pe} + 4\fraccorche{e}{Pa}X_2\corche{Pa} \le 40\corche{e} \\
    8\fraccorche{l}{Pe}X_1\corche{Pe} + 4\fraccorche{l}{Pa}X_2\corche{Pa} \le 40\corche{l} \\
    3\fraccorche{en}{Pe}X_1\corche{Pe} + 3\fraccorche{en}{Pa}X_2\corche{Pa} \le 20\corche{en} \\
\end{align*}
\insertarTabla{02.11.01.tex}{Solución Óptima}
La tabla anterior nos indica que debemos producir 3 y un tercio de relojes de cada tipo, para obtener una ganancia total de \$1667 aproximadamente. En este caso, el recurso limitante resultan ser las horas de ensamblaje.

\subsubsection{Cambio en ganancia Pedestal}
La solución sigue siendo Óptima, obteniendo una ganancia de \$1914,75. \\
\insertarTabla{02.11.02.tex}{Nueva Solución Óptima}
\subsubsection{Y además el de pared}
En este caso, sin embargo, no podemos decir lo mismo en este caso.
\insertarTabla{02.11.03.tex}{Nueva solución NO Óptima}
\subsubsection{Más trabajo}
Según los precios sombra y costos reducidos, sabemos que:
\begin{itemize}
    \item 1 hora extra de David produce $\$0$ extra de ganancia
    \item 1 hora extra de Daiana produce $\$25$ extra de ganancia
    \item 1 hora extra de Lidia produce $\$33.33$ extra de ganancia
\end{itemize}
Por lo tanto, resulta más beneficioso que sea Lidia quien trabaje una hora extra.

\subsubsection{Lidia, Lidia}
En este caso, el aumento de horas de Lidia resulta valido (la solución sigue siendo factible) mejorando nuestra ganancia en $\$166.66$
\begin{align*}
    B^{-1}.b_n = b^{*} \\
    \begin{bmatrix}
        1 & -0.5 & -0.62 \\
        0 & 0.25 & -0.33 \\
        0 & -0.25 & 0.67 
    \end{bmatrix}.
    \begin{bmatrix}
        40 \\
        40 \\
        25
    \end{bmatrix}=
    \begin{bmatrix}
        3.25 \\
        1.68\\
        6.68
    \end{bmatrix}
\end{align*}

\end{homeworkProblem}