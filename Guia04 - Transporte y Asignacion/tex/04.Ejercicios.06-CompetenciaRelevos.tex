
\begin{homeworkProblem}[-1][Competencia Relevos]
A continuación se presentan los tiempos estimados de cada nadador en cada estilo
\insertarTabla{01.06.01.tex}{Datos Iniciales}
Luego, aplicando el método húngaro (se omiten las restas por filas debido a la existencia de dos columnas ficticias)
\insertarTabla{01.06.02.tex}{Resta por columnas}
\insertarTabla{01.06.03.tex}{Primer intento de asignación - restamos $X_{22}=1$}
\insertarTabla{01.06.04.tex}{Segundo intento de asignación restamos $X_{41}=1$}
\insertarTabla{01.06.05.tex}{Asignación Óptima}
La asignación Óptima de nadadores consiste entonces en que el nadador 01 realice la etapa de dorso, el 02 la de Mariposa, el tercero en estilo libre y el quinto en Pecho. Los otros dos nadadores nos conviene que no participan, y cabe destacar que los dos primeros nadadores pueden intercambiar sus asignaciones sin afectar el tiempo total estimado.
\end{homeworkProblem}


