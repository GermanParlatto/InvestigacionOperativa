
\begin{homeworkProblem}[-1][JoShop]

\subsubsection{Asignación Óptima}
\insertarTabla{01.05.01.tex}{Situación Inicial}
\insertarTabla{01.05.02.tex}{Resta por filas}
\insertarTabla{01.05.03.tex}{Resta por columnas}
\insertarTabla{01.05.04.tex}{Primer intento - Restar $X_{24}=10$}
\insertarTabla{01.05.05.tex}{Segundo intento - Restar $X_{41}=10$}
\insertarTabla{01.05.06.tex}{Asignación Óptima}
Gracias a la aplicación del método húngaro, llegamos a que la siguiente asignación del personal traerá acarreado los costos mínimos:
\begin{itemize}
    \item Asignar al trabajador 01 a la tarea 04.
    \item Asignar al trabajador 02 a la tarea 03.
    \item Asignar al trabajador 03 a la tarea 02.
    \item Asignar al trabajador 04 a la tarea 01.
\end{itemize}
Dicho costo resulta ser de 140 unidades monetarias.

\subsubsection{Quinto Trabajador}
\insertarTabla{01.05.07.tex}{+1 Trabajador - Situación Inicial}
\insertarTabla{01.05.08.tex}{+1 Trabajador - Resta por Filas}
\insertarTabla{01.05.09.tex}{+1 Trabajador - Resta por Columnas}
\insertarTabla{01.05.10.tex}{+1 Trabajador - Asignación Óptima}
En este caso la asignación Óptima es:
\begin{itemize}
    \item Asignarle la tarea 01 al trabajador 05 (nuevo trabajador).
    \item Asignarle la tarea 02 al trabajador 04.
    \item Asignarle la tarea 03 al trabajador 02.
    \item Asignarle la tarea 04 al trabajador 01.
\end{itemize}
Dicho costo resulta ser de 120 unidades monetarias.

\subsubsection{Quinta Tarea}
\insertarTabla{01.05.11.tex}{+1 Tarea - Situación Inicial}
\insertarTabla{01.05.12.tex}{+1 Tarea - Resta por Filas}
\insertarTabla{01.05.13.tex}{+1 Tarea - Resta por Columnas}
\insertarTabla{01.05.14.tex}{+1 Tarea - Asignación Óptima}
La asignación Óptima resulta:
\begin{itemize}
    \item Asignar al trabajador 01 a la tarea 04.
    \item Asignar al trabajador 02 a la tarea 03.
    \item Asignar al trabajador 03 a la tarea 05 (nueva tarea).
    \item Asignar al trabajador 04 a la tarea 02.
\end{itemize}
Entonces la nueva tarea debe tener prioridad sobre la tarea 01 ya que caso contrario los costos se disparan. Con un costo de 80 unidades monetarias.
\end{homeworkProblem}


