\begin{homeworkProblem}[-1][Pasta Dental revisited]
Datos:
\begin{itemize}
    \item $D$: 6500.
    \item $k$: 1200.
\end{itemize}
\subsubsection{Costo mínimo anual}
\begin{align*}
    q_1=\sqrt{
        \frac{
            2\times 1200\fraccorche{\$}{Orden}\times 2340000\fraccorche{Kg}{\ano}
            }{
                10.8\fraccorche{\$}{Kg.\ano}\times 1\corche{Orden}
            }
        }
        =
        22803.50\fraccorche{Kg}{Orden}
\end{align*}
\begin{align*}
    q_2=\sqrt{
        \frac{
            2\times 1200\fraccorche{\$}{Orden}\times 2340000\fraccorche{Kg}{\ano}
            }{
                7.2\fraccorche{\$}{Kg.\ano}\times 1\corche{Orden}
            }
        }
        =
        27928.48\fraccorche{Kg}{Orden}
\end{align*}
\begin{align*}
    q_3=\sqrt{
        \frac{
            2\times 1200\fraccorche{\$}{Orden}\times 2340000\fraccorche{Kg}{\ano}
            }{
                5.4\fraccorche{\$}{Kg.\ano}\times 1\corche{Orden}
            }
        }
        =
        32249.03\fraccorche{Kg}{Orden}
\end{align*}
Como vemos que el segundo lote óptimo se encuentra en el rango que le corresponde, calculamos su CTE y lo comparamos con el CTE del límite superior de dicho intervalo:
\begin{align*}
    CTE_{q2} = 1200\fraccorche{\$}{Orden}&\times \frac{2340000\fraccorche{Kg}{\ano}}{27928.48\fraccorche{Kg}{Orden}}  + \frac{1}{2}\times \\ & 27928.48\fraccorche{Kg}{Orden}\times 7.2\fraccorche{\$}{Kg.\ano}\times 1\corche{\ano}=201085.05
\end{align*}
\begin{align*}
    CTE_{Q2=40000} = 1200\fraccorche{\$}{Orden}&\times \frac{2340000\fraccorche{Kg}{\ano}}{40000\fraccorche{Kg}{Orden}}  + \frac{1}{2}\times \\ & 40000\fraccorche{Kg}{Orden}\times 7.2\fraccorche{\$}{Kg.\ano}\times 1\corche{\ano}=214200
\end{align*}

\subsubsection{Cantidad por pedido}
Basándonos en los datos anteriores, resulta conveniente pedir $27938.48$ Kg por Orden ($q_2$)

\subsubsection{Gráficas}
\insertarImagen{03}
\end{homeworkProblem}