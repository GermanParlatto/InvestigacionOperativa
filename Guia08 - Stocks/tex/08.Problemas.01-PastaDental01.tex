\begin{homeworkProblem}[-1][Pasta Dental]
\subsubsection{Cantidad de Jarabe de Sorbitol retirado}
Los datos con los que contamos son:
\begin{itemize}
    \item Demanda $D=5000\fraccorche{Kg}{Dia}$
    \item Costo de Mantenimiento $c_1=0.02\fraccorche{\$}{Kg.Mes}$
    \item Costo de Pedido/Orden $K=1200\fraccorche{\$}{Orden}$
    \item Parámetro de Dimensionamiento $T=1$
\end{itemize}
Entonces:
\begin{align*}
    q_0=\sqrt{
        \frac{
            2\times 1200\fraccorche{\$}{Orden}\times 5000\fraccorche{Kg}{Dia} \times 30\fraccorche{Dia}{Mes}
            }{
                0.02\fraccorche{\$}{Kg.Mes}\times 1\corche{Orden}
            }
        }
        =
        134164.0786\fraccorche{Kg}{Orden}
\end{align*}
La cantidad Óptima a adquirir en cada orden que se le realiza al proveedor es de 134164.08 Kg.

\subsubsection{Reposición de Jarabe}
Sabiendo que $n$ es la cantidad de pedidos en un período $T$, $t$ la duración del período entre reaprovisionamientos, $D$ la demanda del período $T$ y $q_0$ el lote óptimo por orden, entonces:
\begin{align*}
    n=\frac{D}{q_0}=\frac{T}{t} \rightarrow t = \frac{q\times T}{D}
\end{align*}
\begin{align*}
    t = \frac{
            134164.08\fraccorche{Kg}{Orden}\times 1 \corche{Orden}
        }{
            5000\fraccorche{Kg}{Dia}\times 30\fraccorche{Dia}{Mes}
        } =  0.8944\corche{Mes} \approx 27\corche{Dia}       
\end{align*}
Por lo tanto debemos reaprovisionar los stocks de jarabe de sorbitol una vez cada 26/27 días.

\subsubsection{Gastos Anuales}
Sabiendo la composición del costo total esperado:
\begin{align*}
    &CTE& &=& &Compra& &+& &Preparaci\acute on& &+& &Retenci \acute on& \\
    \intertext{Entonces podemos expresar el mismo como:} \\
    &CTE& &=& &\paren{b\times D}& &+& &\paren{k\times \frac{D}{q_0}}& &+& &\paren{\frac{1}{2}\times c_1 \times q_0 \times T}& \\
    \intertext{Reemplazando por los correspondientes valores} \\
    &CTE& &=& &0&
     &+&
      &\paren{ 16099.68\fraccorche{\$}{A\text{ñ}o}}&  &+& 
      &\paren{16099.68\fraccorche{\$}{A\text{ñ}o}}&  \\
\end{align*}


\subsubsection{Gráfica de Costos}
\insertarImagen{04}

\subsubsection{Recortes}
Podríamos cambiar el proveedor del almacénamiento. De esa manera, el valor del costo total disminuiría debido a que la función de costo de almacénamiento poseería una pendiente más cercana al eje X.

\subsubsection{Mayor Demanda de Sorbitol}
\begin{align*}
    q_0=\sqrt{
        \frac{
            2\times 1200\fraccorche{\$}{Orden}\times \paren{5000\times 1.3}\fraccorche{Kg}{Dia} \times 30\fraccorche{Dia}{Mes}
            }{
                0.02\fraccorche{\$}{Kg.Mes}\times 1\corche{Orden}
            }
        }
        =
        152970.58\fraccorche{Kg}{Orden}
\end{align*}
\begin{align*}
    CTE = 1200\fraccorche{\$}{Orden}&\times \frac{\paren{1800000\times 1.3}\fraccorche{Kg}{Mes}}{152970.58\fraccorche{Kg}{Orden}} \\ &+ 0.5\times 152970.58\fraccorche{Kg}{Orden}\times 0.24\fraccorche{\$}{Kg.A\text{ñ}o}=36712.94
\end{align*}

\subsubsection{Otra Estrategia}
Si se mantiene el primer valor calculado de lote óptimo:
\begin{align*}
    CTE = 1200\fraccorche{\$}{Orden}&\times \frac{\paren{1800000\times 1.3}\fraccorche{Kg}{Mes}}{134164.08\fraccorche{Kg}{Orden}} \\ &+ 0.5\times 134164.08\fraccorche{Kg}{Orden}\times 0.24\fraccorche{\$}{Kg.A\text{ñ}o}=37029.28
\end{align*}
Entonces:
\begin{align*}
    \frac{37029.28}{36712.94}=1.008
\end{align*}
Por lo que no modificar la política nos lleva a incurrir en un costo de casi un 1\% más.
\end{homeworkProblem}