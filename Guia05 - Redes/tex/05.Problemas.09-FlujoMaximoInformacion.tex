\begin{homeworkProblem}[-1][Flujo Maximo de Informacion]

\centerline{Inicial}
\begin{figure}[h]
    \centering
    \begin{adjustbox}{max width=\textwidth, max height=0.35\textheight}
        \begin{tikzpicture}[
            transform shape,
            ->, >=stealth',
            auto, node distance=3cm,
            thick,
            main node/.style={
                circle, draw, font=\sffamily\Large\bfseries
            }
        ]

        \node[main node] (n1) at (-2,00)  {1};
        \node[main node] (n2) at (03,03)  {2};
        \node[main node] (n3) at (03,-3)  {3};
        \node[main node] (n4) at (09,-3)  {4};
        \node[main node] (n5) at (09,03)  {5};
        \node[main node] (n6) at (14,00)  {6};

        \foreach \from/\to/\te/\out/\in in {
            n1/n2/15/65/180,n2/n1/0/180/65,
            n1/n3/18/295/180,n3/n1/0/180/295,
            n2/n5/12/0/180,n5/n2/7/180/0,
            n2/n3/10/270/90,n3/n2/0/90/270,
            n3/n5/6/45/225,n5/n3/0/225/45,
            n3/n4/11/0/180,n4/n3/0/180/0,
            n3/n6/15/305/290,n6/n3/0/290/305,
            n4/n5/26/90/270,n5/n4/14/270/90,
            n4/n6/12/10/250,n6/n4/0/250/10,
            n5/n6/9/350/110,n6/n5/0/110/350
        }{
            \draw [-] (\from)  [out=\out, in=\in] to node[very near start] {$\te$} (\to) ;
        }
        \end{tikzpicture}
    \end{adjustbox}
\end{figure}
\FloatBarrier

\centerline{Trayectoria de Aumento 01: $1-(15)-2-(12)-5-(9)-6\ =\ 9$}
\begin{figure}[h]
    \centering
    \begin{adjustbox}{max width=\textwidth, max height=0.35\textheight}
        \begin{tikzpicture}[
            transform shape,
            ->, >=stealth',
            auto, node distance=3cm,
            thick,
            main node/.style={
                circle, draw, font=\sffamily\Large\bfseries
            }
        ]

        \node[main node] (n1) at (-2,00)  {1};
        \node[main node] (n2) at (03,03)  {2};
        \node[main node] (n3) at (03,-3)  {3};
        \node[main node] (n4) at (09,-3)  {4};
        \node[main node] (n5) at (09,03)  {5};
        \node[main node] (n6) at (14,00)  {6};

        \foreach \from/\to/\te/\out/\in in {
            n1/n2/6/65/180,n2/n1/9/180/65,
            n1/n3/18/295/180,n3/n1/0/180/295,
            n2/n5/3/0/180,n5/n2/16/180/0,
            n2/n3/10/270/90,n3/n2/0/90/270,
            n3/n5/6/45/225,n5/n3/0/225/45,
            n3/n4/11/0/180,n4/n3/0/180/0,
            n3/n6/15/305/290,n6/n3/0/290/305,
            n4/n5/26/90/270,n5/n4/14/270/90,
            n4/n6/12/10/250,n6/n4/0/250/10,
            n5/n6/0/350/110,n6/n5/9/110/350
        }{
            \draw [-] (\from)  [out=\out, in=\in] to node[very near start] {$\te$} (\to) ;
        }
        \end{tikzpicture}
    \end{adjustbox}
\end{figure}
\FloatBarrier
\pagebreak


\centerline{Trayectoria de Aumento 02: $1-(6)-2-(10)-3-(6)-5-(14)-4-(12)-6\ =\ 6$}
\begin{figure}[h]
    \centering
    \begin{adjustbox}{max width=\textwidth, max height=0.35\textheight}
        \begin{tikzpicture}[
            transform shape,
            ->, >=stealth',
            auto, node distance=3cm,
            thick,
            main node/.style={
                circle, draw, font=\sffamily\Large\bfseries
            }
        ]

        \node[main node] (n1) at (-2,00)  {1};
        \node[main node] (n2) at (03,03)  {2};
        \node[main node] (n3) at (03,-3)  {3};
        \node[main node] (n4) at (09,-3)  {4};
        \node[main node] (n5) at (09,03)  {5};
        \node[main node] (n6) at (14,00)  {6};

        \foreach \from/\to/\te/\out/\in in {
            n1/n2/0/65/180,n2/n1/15/180/65,
            n1/n3/18/295/180,n3/n1/0/180/295,
            n2/n5/3/0/180,n5/n2/16/180/0,
            n2/n3/4/270/90,n3/n2/6/90/270,
            n3/n5/0/45/225,n5/n3/6/225/45,
            n3/n4/11/0/180,n4/n3/0/180/0,
            n3/n6/15/305/290,n6/n3/0/290/305,
            n4/n5/32/90/270,n5/n4/8/270/90,
            n4/n6/6/10/250,n6/n4/6/250/10,
            n5/n6/0/350/110,n6/n5/9/110/350
        }{
            \draw [-] (\from)  [out=\out, in=\in] to node[very near start] {$\te$} (\to) ;
        }
        \end{tikzpicture}
    \end{adjustbox}
\end{figure}
\FloatBarrier

\centerline{Trayectoria de Aumento 03: $1-(18)-3-(15)-6\ =\ 15$}
\begin{figure}[h]
    \centering
    \begin{adjustbox}{max width=\textwidth, max height=0.35\textheight}
        \begin{tikzpicture}[
            transform shape,
            ->, >=stealth',
            auto, node distance=3cm,
            thick,
            main node/.style={
                circle, draw, font=\sffamily\Large\bfseries
            }
        ]

        \node[main node] (n1) at (-2,00)  {1};
        \node[main node] (n2) at (03,03)  {2};
        \node[main node] (n3) at (03,-3)  {3};
        \node[main node] (n4) at (09,-3)  {4};
        \node[main node] (n5) at (09,03)  {5};
        \node[main node] (n6) at (14,00)  {6};

        \foreach \from/\to/\te/\out/\in in {
            n1/n2/0/65/180,n2/n1/15/180/65,
            n1/n3/3/295/180,n3/n1/15/180/295,
            n2/n5/3/0/180,n5/n2/16/180/0,
            n2/n3/4/270/90,n3/n2/6/90/270,
            n3/n5/0/45/225,n5/n3/6/225/45,
            n3/n4/11/0/180,n4/n3/0/180/0,
            n3/n6/0/305/290,n6/n3/15/290/305,
            n4/n5/32/90/270,n5/n4/8/270/90,
            n4/n6/6/10/250,n6/n4/6/250/10,
            n5/n6/0/350/110,n6/n5/9/110/350
        }{
            \draw [-] (\from)  [out=\out, in=\in] to node[very near start] {$\te$} (\to) ;
        }
        \end{tikzpicture}
    \end{adjustbox}
\end{figure}
\FloatBarrier
\pagebreak

\centerline{Trayectoria de Aumento 04: $1-(18)-3-(15)-4-()-6\ =\ 3$}
\begin{figure}[h]
    \centering
    \begin{adjustbox}{max width=\textwidth, max height=0.35\textheight}
        \begin{tikzpicture}[
            transform shape,
            ->, >=stealth',
            auto, node distance=3cm,
            thick,
            main node/.style={
                circle, draw, font=\sffamily\Large\bfseries
            }
        ]

        \node[main node] (n1) at (-2,00)  {1};
        \node[main node] (n2) at (03,03)  {2};
        \node[main node] (n3) at (03,-3)  {3};
        \node[main node] (n4) at (09,-3)  {4};
        \node[main node] (n5) at (09,03)  {5};
        \node[main node] (n6) at (14,00)  {6};

        \foreach \from/\to/\te/\out/\in in {
            n1/n2/0/65/180,n2/n1/15/180/65,
            n1/n3/0/295/180,n3/n1/18/180/295,
            n2/n5/3/0/180,n5/n2/16/180/0,
            n2/n3/4/270/90,n3/n2/6/90/270,
            n3/n5/0/45/225,n5/n3/6/225/45,
            n3/n4/8/0/180,n4/n3/3/180/0,
            n3/n6/0/305/290,n6/n3/15/290/305,
            n4/n5/32/90/270,n5/n4/8/270/90,
            n4/n6/3/10/250,n6/n4/9/250/10,
            n5/n6/0/350/110,n6/n5/9/110/350
        }{
            \draw [-] (\from)  [out=\out, in=\in] to node[very near start] {$\te$} (\to) ;
        }
        \end{tikzpicture}
    \end{adjustbox}
\end{figure}
\FloatBarrier

A continuacion, la red de flujo maximo (igual a 33)
\begin{figure}[h]
    \centering
    \begin{adjustbox}{max width=\textwidth, max height=0.35\textheight}
        \begin{tikzpicture}[
            transform shape,
            ->, >=stealth',
            auto, node distance=3cm,
            thick,
            main node/.style={
                circle, draw, font=\sffamily\Large\bfseries
            }
        ]

        \node[main node] (n1) at (-2,00)  {1};
        \node[main node] (n2) at (03,03)  {2};
        \node[main node] (n3) at (03,-3)  {3};
        \node[main node] (n4) at (09,-3)  {4};
        \node[main node] (n5) at (09,03)  {5};
        \node[main node] (n6) at (14,00)  {6};

        \foreach \from/\to/\te/\out/\in in {
            n1/n2/15/65/180,
            n1/n3/18/295/180,
            n2/n5/16/50/130,n5/n2/3/180/0,
            n2/n3/6/270/90,
            n3/n5/6/45/225,
            n3/n4/3/0/180,
            n3/n6/15/305/290,
            n4/n5/8/60/300,n5/n4/32/240/120,
            n4/n6/9/10/250,
            n5/n6/9/350/110
        }{
            \draw [->] (\from)  [out=\out, in=\in] to node[blue] {$\te$} (\to) ;
        }
        \end{tikzpicture}
    \end{adjustbox}
\end{figure}
\FloatBarrier
\pagebreak
\end{homeworkProblem}