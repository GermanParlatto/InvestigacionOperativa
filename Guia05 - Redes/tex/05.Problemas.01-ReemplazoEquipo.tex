\begin{homeworkProblem}[-1][Reemplazo de Equipos]

%\tikzstyle{vertex}=[auto=left,circle,fill=black!25,minimum size=20pt,inner sep=0pt]
Problema Lineal:
\begin{align*}
    Min\ Z = 4000X_{12} + &4300X_{23} + 4800X_{34} + 4900X_{45} \\ + &5400X_{13} 
    + 9800X_{14} + 6200X_{24} + 8700X_{25} + 7100X_{35}
\end{align*}
Sujeto a:
\begin{align*}
    &\sum_{j=1}^{n-1}{X_{1j}}=1 \\
    &\sum_{i=1}^{n-1}{X_{i5}}=1 \\
    &\sum_{i=1}^{5}{X_{ik}} = \sum_{j=1}^{5}{X_{kj}}\quad\quad
    \forall k \neq 1,5 \\
    &\forall i,j \in \left\{1,2,3,4,5\right\} X_{ij} \in \left\{0,1\right\}
\end{align*}

\begin{figure}[h]
    \centering
    \begin{adjustbox}{max width=\textwidth}
        \begin{tikzpicture}[
            transform shape,
            ->, >=stealth',
            auto, node distance=3cm,
            thick,
            main node/.style={
                circle, draw, font=\sffamily\Large\bfseries
            }
        ]

        \node[main node] (n1) at (0,0)  {1};
        \node[main node] (n2) at (4,0)  {2};
        \node[main node] (n3) at (8,0) {3};
        \node[main node] (n4) at (12,0) {4};
        \node[main node] (n5) at (16,0) {5};

        \foreach \from/\to/\te/\out\in in {
            n1/n2/4000/0/180,
            n1/n3/5400/30/150,
            n1/n4/9800/330/210,
            n2/n3/4300/0/180,
            n2/n4/6200/30/150,
            n2/n5/8700/330/210,
            n3/n4/4800/0/180,
            n3/n5/7100/30/150,
            n4/n5/4800/0/180
        }{
            \draw [->] (\from)  [out=\out, in=\in] to node {$\te$} (\to) ;
        }
        \end{tikzpicture}
    \end{adjustbox}
    \caption{Grafo 01}
\end{figure}

\insertarTabla{01.01.01.tex}{Iteraciónes del Metodo}
En la tabla anterior se aprecia que la ruta más corta es $1\rightarrow 3\rightarrow 5$, lo que significa que nos conviene adquirir el equipo en el primer año y renovarlo cada 2 años.

\end{homeworkProblem}