\begin{homeworkProblem}[-1][Quilmes]

\subsubsection{Programa Lineal}
Variables de Decisión:
\begin{itemize}
    \item $x$: cantidad de dinero a invertir en publicidad en Buenos Aires.
    \item $y$: cantidad de dinero a invertir en publicidad en el Interior.
\end{itemize}
Función Objetivo:
\begin{align*}
    Max\ Z = 30\sqrt{x} + 20\sqrt{y}
\end{align*}
Sujeta a:
\begin{align*}
    x+y\leq 100 \\ x,y\geq0
\end{align*}

Para lograr maximizar los beneficios obtenidos, recomendamos la inversión de $\$69.23$ en publicidad destinada a Buenos Aires y el resto ($\$30.76$) al interior. Con esto, se alcanzaria un ingreso de $\$360.55$.

\subsubsection{Condiciónes de Óptimo}

\begin{align*}
    \frac{\partial f}{\partial x} - \lambda \frac{\partial g}{\partial x} \leq 0 \rightarrow \frac{15}{\sqrt{69.23}} - \lambda \leq 0 \\
    \frac{\partial f}{\partial y} - \lambda \frac{\partial g}{\partial y} \leq 0 \rightarrow \frac{10}{\sqrt{30.77}} - \lambda \leq 0 \\
\end{align*}
Ya que tanto $x$ como $y$ son distintas de cero, las expresiónes anteriores son iguales a $0$, por lo que entonces:
\begin{align*}
    \lambda = 1.80
\end{align*}
Ademas, en el punto candidato a solución Óptima:
\begin{align*}
    \lambda\corche{g(x)-b} = 0 \rightarrow \lambda\corche{x+y-100} = 0
\end{align*}

Debido a lo anterior, podemos asegurar que se cumplen las condiciónes de Óptimalidad.



\subsubsection{Un peso más...}
Teniendo en cuenta el valor de $\lambda$ el cual representa que tanto mejoraria nuestro funciónal por unidad monetaria extra que se disponga
podemos asegurar que la ganancia, de disponer de un peso más detinado a inversión en pubilicidad, aumentaria en $\$1.8$ extra.
\end{homeworkProblem}