\begin{homeworkProblem}[-1][Publicidad]

\subsubsection{Programa Lineal}
Variables de Decisión:
\begin{itemize}
    \item $x$: cantidad de minutos a comprar de comerciales en tv.
    \item $y$: cantidad de minutos a comprar de comerciales en radio.
\end{itemize}
Función Objetivo:
\begin{align*}
    Max\ Z = -2x^2-y^2+xy+8x+3y
\end{align*}
Sujeta a:
\begin{align*}
    3000x+1000y\leq 10000 \\ x,y\geq0
\end{align*}
\subsubsection{Solucion}
Para obtener el mejor rendimiento del dinero destinado a publicidad, recomendamos que la compañía adquiera $2.46$ minutos de comerciales en TV y $2.6$ en radio. De seguir este plan, los ingresos obtenidos ascenderían a los $15017.86$

\textbf{Condición Necesaria:}\\
Aplicamos el método de los multiplicadores de Lagrange.
\begin{align*}
    &L(x,y,\lambda)=-2x^2-y@+xy+8x+3y+\lambda\paren{10000-3000x-1000y}\\
    &\nabla L = \paren{-x+y+8-3000\lambda;-2y+x+3-1000\lambda ;10000 - 3000x-1000y} = (0;0;0)\\
    &x = \frac{69}{28} \\
    &y = \frac{73}{28} \\
    &\lambda = \frac{1}{4000}
\end{align*}


\textbf{Condición Suficiente:}\\
\begin{align*}
    \mathcal{H}\paren{\frac{69}{28},\frac{73}{28},\frac{1}{4000}}=
    \large{\begin{bmatrix}[2]
        0 & \frac{\partial g}{\partial x}  & \frac{\partial g}{\partial y} \\
         \frac{\partial g}{\partial x}  & \frac{\partial^2 L}{\partial^2 x} & \frac{\partial^2 L}{\partial x \partial y} \\
         \frac{\partial g}{\partial y} & \frac{\partial^2 L}{\partial y \partial x}& \frac{\partial^2 L }{\partial^2 y } 
    \end{bmatrix}} = \begin{bmatrix}
       0 & 3000 & 1000 \\
       3000 & -4 & 1 \\
       1000 & 1 & -2
    \end{bmatrix}
\end{align*}
Analizando los determinantes del Hessiano Orlado, observamos que la condición suficiente de óptimo local es verificada por el punto hallado anteriormente, con lo que podemos asegurar entonces que el mismo es un Máximo.

\subsubsection{Gastar \$1000 extras?}
Teniendo en cuenta el valor de $\lambda$ el cual representa qué tanto mejoraría nuestro funciónal por unidad monetaria extra que se disponga,
nos inclinamos a des aconsejar la compra de minutos de publicidad extra, ya que de hacer esto, si bien la ganancia aumentaría, la inversión necesaria para obtener esta mejora séria mayor que dicha diferencia.

\end{homeworkProblem}