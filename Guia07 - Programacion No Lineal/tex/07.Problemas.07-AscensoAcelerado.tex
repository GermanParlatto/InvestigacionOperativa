\begin{homeworkProblem}[-1][Ascenso Acelerado]
Siendo la funcion multivariable:
\begin{align*}
    z(x_1,x_2)=-\paren{x_1-\sqrt{5}}^2 -\paren{x_2-\pi}^2 - 10
\end{align*}

\subsubsection{Ascenso Acelerado}   

Siendo el gradiente de la funcion:
\begin{align*}
    \nabla z(x_1,x_2)=\paren{-2x_1+2\sqrt{5};-2x_2+2\Pi}
\end{align*}
Y tomando el punto inicial $X_0=(2;3)$.\\

\textbf{Iteracion 01}\\
\begin{align*}
    &X_1=\paren{2+0.47\lambda ; 3+0.28\lambda}\\
    &z(X_1)=-10.076 + 0.3012\lambda + 0.3\lambda^2\\
    &\frac{\partial z'(X_1) }{\partial \lambda}=-0.6\lambda + 0.3012 = 0 \rightarrow \lambda = 0.5\\
    &X_1(0.5)=\paren{ 2.235 ; 3.14 }\\
    &||\nabla z(X_1)|| = 3.836x10^{-3} \leq 0.05
\end{align*}

Por lo tanto, con un error de $3.836x10^{-3}$ se puede afirmar que existe un maximo en $X=\paren{2.235;3.14}$.


\subsubsection{Newton Raphrson}   

Siendo el gradiente de la funcion:
\begin{align*}
    \nabla z(x_1,x_2)=\paren{-2x_1+2\sqrt{5};-2x_2+2\Pi}
\end{align*}
Y tomando el punto inicial $X_0=(2;3)$.\\

\textbf{Iteracion 01}\\
\begin{align*}
\mathcal{H}_z=\begin{bmatrix}
    -2 & 0 \\ 0 & -2
\end{bmatrix} 
\rightarrow
\mathcal{H}_z^{-1}(x_0)=\begin{bmatrix}
    -\frac{1}{2} & 0 \\ 0 & -\frac{1}{2}
\end{bmatrix} \\
X_1=\begin{bmatrix}
    2 \\ 3
\end{bmatrix}-\begin{bmatrix}
    -\frac{1}{2} & 0 \\ 0 & -\frac{1}{2}
\end{bmatrix}\begin{bmatrix}
     -4+2\sqrt(5) \\ -6 + 2\pi
\end{bmatrix}=\begin{bmatrix}
    \sqrt{5} \\ \pi
\end{bmatrix}
\end{align*}

Por lo tanto, se puede afirmar que existe un maximo en $X=\paren{\sqrt{5};\pi}$.



\end{homeworkProblem}