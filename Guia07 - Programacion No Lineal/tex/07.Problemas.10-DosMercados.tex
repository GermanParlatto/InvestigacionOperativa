\begin{homeworkProblem}[-1][Dos mercados]

\subsubsection{Problema Lineal}
Funcion Objetivo:
\begin{align*}
    Max\ Z = \paren{60q_1+\paren{100-q_2}q_2} - \paren{q_1+q_2}^2
\end{align*}
\subsubsection{Solucion Iterativa}

Utilizando la siguiente expresion como paso general:
\begin{align*}
    X^{k+1}=X^k-\lambda^k.\nabla z\paren{X^k}
\end{align*} 
Siendo el gradiente de la funcion:
\begin{align*}
    \nabla z(q_1,q_2)=\paren{60-2q_1-2q_2;100-4q_2-2q_1}
\end{align*}
Y tomando el punto inicial $X_0=(5;10)$ con un error $\varepsilon=0.1$.\\

\textbf{Iteracion 01}\\
\begin{align*}
    &X^{1}=X^0-\lambda^0.\nabla z\paren{X^0}=\paren{5+30\lambda;10+50\lambda}\\
    &z(X^1)=60(5+30\lambda)+(100-(10+50\lambda))(10+50\lambda)-((5+3\lambda)+(10+50\lambda))^2 \\
    &\frac{\partial z'(X^1) }{\partial \lambda}=
        -17800\lambda +300 = 0 
        \rightarrow \lambda = \frac{17}{89}\\
    &X^1\paren{\frac{17}{89}}=\paren{\frac{955}{89};\frac{1740}{89}}\\
    &||\nabla z(X^1)|| = 0.655 \geq \varepsilon
\end{align*}

\textbf{Iteracion 02}\\
\begin{align*}
    &X^{2}=X^1-\lambda^1.\nabla z\paren{X^1}=\paren{\frac{955}{89}-\frac{50}{89}\lambda;\frac{1740}{89}+\frac{30}{89}\lambda}\\
    &z(X^2)=\frac-{1300}{7921}\lambda^2+\frac{3400}{7921}\lambda+\frac{115675}{89} \\
    &\frac{\partial z'(X^2) }{\partial \lambda}=
        \frac-{2600}{7921}\lambda+\frac{3400}{7921}= 0 
        \rightarrow \lambda = \frac{17}{13}\\
    &X^2\paren{\frac{17}{13}}=\paren{\frac{11565}{1157};\frac{23130}{1157}}\\
    &||\nabla z(X^2)|| = 0.05042 \leq \varepsilon
\end{align*}

Por lo tanto, con un error de $0.05042$ se puede afirmar que existe un maximo en $X=\paren{\frac{11565}{1157};\frac{23130}{1157}}$.


\subsubsection{Solucion Analitica}
Condicion Necesaria:
\llavesR{
    \frac{\partial z}{\partial q_1}= 0 \rightarrow 60-2q_1-2q_2=0 \\
    \frac{\partial z}{\partial q_2}= 0 \rightarrow 100-4q_2-2q_1=0 
}{
    q_1=10 \wedge q_2=20
}
Condicion Suficiente:\\
\begin{align*}
    \mathcal{H}=\large{\begin{bmatrix}[2]
        \frac{\partial^2 z}{\partial q_1^2}  & \frac{\partial^2 z}{\partial q_1q_2} \\
         \frac{\partial^2 z}{\partial q_2q_1} & \frac{\partial^2 z}{\partial q_2^2}
    \end{bmatrix}} = \begin{bmatrix}
        -2 & -2 \\ -2 & -4
    \end{bmatrix}
\end{align*}
A partir del analisis de los determinantes de la hessiana, comproobamos que la condicion de óptimo local se cumple, por lo que podemos assegurar encontrarnos frente a un maximo.
\end{homeworkProblem}